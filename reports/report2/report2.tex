\documentclass[10pt,twocolumn,letterpaper]{article}

\usepackage{cvpr}
\usepackage[utf8]{inputenc}
\usepackage{gensymb}
\usepackage{graphicx}
% todo: change the image folder
\graphicspath{ {report2-imgs/} }
\usepackage{float}

\cvprfinalcopy
\def\cvprPaperID{a1700831}


% begin of document
\begin{document}
\title{Assignment 2 - Method for Face Detection by Viola and Jones}
\author{Yuanzhong Xia\\
University of Adelaide\\
SA, Australia\\
{\tt\small a1700831@student.adelaide.edu.au}
}
\maketitle

% abstract
\begin{abstract}
This assignment 2 report describes the Viola and Jones's method{\cite{origin}} which can detect a specified object very rapidly.
Their method uses a machine learning method named AdaBoost to both select 
% todo !!!!!!!!!!!!

\end{abstract}


\section{The Problem}
The problem is to detect face(s) in an input image, and output the image with each face marked with a bounding box.
Although it looks very simple and we don't need to recognize a specific face, the challenge is to detect all the faces in real time.
Because in any normal image, there are so many possible positions and sizes for a single face,
and it's impossible to try every sub-window of them due to the huge computational cost.

Moreover, an input image is not always predictable. The head direction, face integrity, environmental light brightness,
camera features, image quality, etc. can often bring issues to the face detection problem.
And to simplify the problem, the method from Viola and Jones demonstrates with a frontal face detection system,
which means the effects of ``head direction'' and ``face integrity'' are ignored.

\section{Background}
% todo: describe the competing approaches to the problem


\section{Viola and Jones's Method}

% the OpenCV implementation, the different models
% the training data is unknown although I searched for XXXXXX

\section{Hypothesis}
% \subsection{Limitation and Improvements}
% https://blog.bluecrossmn.com/future-face-minnesota-look-willmar/ - the picture source
% http://web01a.elitedaily.com/envision/photographer-brilliantly-captures-the-many-different-facial-expressions-we-all-go-through-photos/

The performance is fully depending on the training data.

The prediction result is very stable, using the same model to predict will always get the same result.

% how to come over the slightly tilted face (the Haar basis features can do that)
In principle, any kind of face can be detected if the training data have this 

\section{Experiments}

\section{Conclusion}

% Bibliography
\begin{thebibliography}{9}
\bibitem {origin}
P. Viola and M. Jones, ``Rapid object detection using a boosted cascade of simple features,''
\textit{Proc. IEEE Conf. Comput. Vis. Pattern Recog.}, 2001, pp. 511–518.


\end{thebibliography}

\end{document}

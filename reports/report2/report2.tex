\documentclass[10pt,twocolumn,letterpaper]{article}

\usepackage{cvpr}
\usepackage[utf8]{inputenc}
\usepackage{gensymb}
\usepackage{graphicx}
% TODO: change the image folder
\graphicspath{ {report2-imgs/} }
\usepackage{float}

\cvprfinalcopy
\def\cvprPaperID{a1700831}


% begin of document
\begin{document}
\title{Assignment 2 - Method for Face Detection by Viola and Jones}
\author{Yuanzhong Xia\\
University of Adelaide\\
SA, Australia\\
{\tt\small a1700831@student.adelaide.edu.au}
}
\maketitle

% abstract
\begin{abstract}
This assignment 2 report describes the Viola and Jones's method{\cite{origin}} which can detect a specified object very rapidly.
Their method uses a machine learning method named AdaBoost \cite{adaboost} to both select
% TODO: !!!!!!!!!!!!

\end{abstract}


\section{The Problem}
The problem is to detect face(s) in an input image, and output the image with each face marked with a bounding box.
Although it looks very simple and we don't need to recognize a specific face, the challenge is to detect all the faces in real time.
Because in any normal image, there are so many possible positions and sizes for a single face,
and it's impossible to try every sub-window of them due to the huge computational cost.

Moreover, an input image is not always predictable. The head direction, face integrity, environmental light brightness,
camera features, image quality, etc. can often bring issues to the face detection problem.
And to simplify the problem, the method from Viola and Jones demonstrates with a frontal face detection system,
which means the effects of ``head direction'' and ``face integrity'' are ignored.

\section{Background} \label{sec:bg}
% TODO: describe the competing approaches to the problem


\section{Viola and Jones's Method}
Viola and Jones's method is designed to be applied in any object detection case, and the detection speed is extremely fast around the publish year (2001).
In the paper, all details of the algorithm are demonstrated in doing face detection.
Comparing to other face detection methods whose speeds are faster than Viola and Jones's method,
those methods' speeds are based on checking image difference between the adjacent frames,
while Viola and Jones's method in the demo proceeds each frame independently which means it can achieve even higher speed using the frame difference.

The method has three main steps:
\begin{enumerate}
    \item Compute integral image and Harr-like features;
    \item Construct a classifier by selecting important features using AdaBoost;
    \item Combine classifiers complex classifiers into a cascading structure;
\end{enumerate}

The final cascading classifiers are the final model, classifiers in each layer can reject most features.
Therefore, the computing time gets reduced significantly.

\subsection{Computing features}
The features used in this paper are Haar basis functions motivated by Papageorgiou et al.'s work. (mentioned in Section \ref{sec:bg})
The features are represented in a rectangle split into two, three or four. Then, divide the sub-rectangles into two classes: A and B.
The feature is calculated by the difference of the sum of all pixels marked under class A and the sum of all pixels marked under class B.

To calculate the features, the authors introduce an ``integral image'', which is an array with the same size of the original image ($width \times height$).
Unlike the image pixel array, each cell of the integral image array stores the accumulated pixel sum from top left to current pixel in original image.

Having integral image, the features can be calculated easily in $O(width \times height)$ time,
each sub-rectangle can be easily calculated by subtracting related integral image cells.
In the paper, the rectangles are tried in each multiply of 24x24 area (i.e. 24x24, 48x48, etc. areas in input images).
Therefore, the exhaustive search set is super large even for a normal images. (For a 384x288 image, the set contains around 180,000 features.)

\subsection{Constructing classifiers}


\subsection{Building cascading classifiers}


\section{Hypothesis}
% \subsection{Limitation and Improvements}
% https://blog.bluecrossmn.com/future-face-minnesota-look-willmar/ - the picture source
% http://web01a.elitedaily.com/envision/photographer-brilliantly-captures-the-many-different-facial-expressions-we-all-go-through-photos/

The performance is fully depending on the training data.

The prediction result is very stable, using the same model to predict will always get the same result.

% TODO: how to come over the slightly tilted face (the Haar basis features can do that)
In principle, any kind of face can be detected if the training data have this

% TODO: the OpenCV implementation, the different models
% TODO: the training data is unknown although I searched for XXXXXX

\section{Experiments}

\section{Extensions}
% TODO: the adjacent frames difference used in video face detection

%

\section{Conclusion}

% Bibliography
\begin{thebibliography}{9}
\bibitem {origin}
P. Viola and M. Jones, ``Rapid object detection using a boosted cascade of simple features,''
In \textit{Proc. IEEE Conf. Comput. Vis. Pattern Recog.}, 2001, pp. 511–518.

\bibitem {adaboost}
Y. Freund and R.E. Schapire, ``A decision-theoretic generalization of on-=line learning and an application to boosting,''
In \textit{Computational Learning Theory: Eurocolt '95}, pages 23-27, Springer-Verlag, 1995.


\end{thebibliography}

\end{document}

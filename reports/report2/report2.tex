\documentclass[10pt,twocolumn,letterpaper]{article}

\usepackage{cvpr}
\usepackage[utf8]{inputenc}
\usepackage{gensymb}
\usepackage{graphicx}
% todo: change the image folder
\graphicspath{ {report2-imgs/} }
\usepackage{float}

\cvprfinalcopy
\def\cvprPaperID{a1700831}


% begin of document
\begin{document}
\title{Assignment 2 - Method for Face Detection by Viola and Jones}
\author{Yuanzhong Xia\\
University of Adelaide\\
SA, Australia\\
{\tt\small a1700831@student.adelaide.edu.au}
}
\maketitle

% abstract
\begin{abstract}
This assignment 2 report describes the Viola and Jones's method{\ref{origin}} which can detect a specified object very rapidly.
Their method uses a machine learning method named AdaBoost to both select 


\end{abstract}


\section{The Problem}

\section{Background}
% todo: describe the competing approaches to the problem

\section{Viola and Jones's Method}

% the OpenCV implementation, the different models

\section{Hypothesis}
% \subsection{Limitation and Improvements}
% https://blog.bluecrossmn.com/future-face-minnesota-look-willmar/ - the picture source

The performance is fully depending on the training data.

The prediction result is very stable, using the same model to predict will always get the same result.

% how to come over the slightly tilted face (the Haar basis features can do that)
In principle, any kind of face can be detected if the training data have this 

\section{Experiments}

\section{Conclusion}

% Bibliography
\begin{thebibliography}{9}
\bibitem {origin}
P. Viola and M. Jones, ``Rapid object detection using a boosted cascade of simple features,''
\textit{Proc. IEEE Conf. Comput. Vis. Pattern Recog.}, 2001, pp. 511–518.


\end{thebibliography}

\end{document}

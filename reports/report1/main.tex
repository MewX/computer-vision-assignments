\documentclass[10pt,twocolumn,letterpaper]{article}

\usepackage{cvpr}
\usepackage[utf8]{inputenc}

\cvprfinalcopy
\def\cvprPaperID{a1700831} % *** Enter the CVPR Paper ID here


% begin of document
\begin{document}
\title{Assignment 1 - Method for Creating Mosaics by Brown and Lowe}
\author{Yuanzhong Xia\\
University of Adelaide\\
SA, Australia\\
{\tt\small a1700831@student.adelaide.edu.au}
}
\maketitle

% abstract
\begin{abstract}
   The ABSTRACT is to be in fully-justified italicized text, at the top
   of the left-hand column, below the author and affiliation
   information. Use the word ``Abstract'' as the title, in 12-point
   Times, boldface type, centered relative to the column, initially
   capitalized. The abstract is to be in 10-point, single-spaced type.
   Leave two blank lines after the Abstract, then begin the main text.
   Look at previous CVPR abstracts to get a feel for style and length.
\end{abstract}


\section{The Problem}
This is the description of the problem \cite{origin}.


\section{Brown and Lowe's Method}
This is the background.

\subsection{Description}
This is the description.

\subsection{Limitation and Improvements}
The limitation and improvements.


\section{Testing}
Hypothesis and testing, and resulting.


% Bibliography
\begin{thebibliography}{9}
\bibitem {origin}
M. Brown and D. Lowe, ``Automatic panoramic image stitching using invariant features,''
\textit{Int. J. Comput. Vision}, vol. 74, no. 1, pp. 59–73, Aug. 2007.

\end{thebibliography}

\end{document}

\documentclass[10pt,twocolumn,letterpaper]{article}

\usepackage{cvpr}
\usepackage[utf8]{inputenc}
\usepackage{gensymb}
\usepackage{graphicx}
\usepackage{caption}
\usepackage{subcaption}
\usepackage{mathtools}
\graphicspath{ {report3-imgs/} }
\usepackage{float}
\PassOptionsToPackage{hyphens}{url}\usepackage{hyperref}

\cvprfinalcopy
\def\cvprPaperID{a1700831}


% begin of document
\begin{document}
\title{Assignment 3 - Method for Generating Image Descriptions}
\author{Yuanzhong Xia\\
University of Adelaide\\
SA, Australia\\
{\tt\small a1700831@student.adelaide.edu.au}
}
\maketitle

% abstract
\begin{abstract}
% TODO: abstract
\end{abstract}

% content
\section{The Problem}
The problem is to give a sentence in natural language of the description for an input image.

This problem is very typical, because it is very challenging that it transforms information from ``image'' type to ``text'' type.
The input image should firstly be understood by the artificial intelligence,
then it should be able to output a descriptive text to tell what the input image shows.

The relative applications can be various, like: helping human to automatically pointing out the objectives and events in a massive number of images,
and helping visually impaired people to see. But the most difficult challenge is not only understanding both image and natural language,
but also the information transformation.


\section{Background}
Before the paper by Karpathy \textit{et al.} \cite{origin}. There are plenty of researches challenging this problem.


\section{Algorithm Description}


\section{Hypothesis}
% 1. two data model problem: one object in one model, not in another model (PS an object)
% 2. input generation size not enough, use small object with high resolution


\section{Experiments}
% details set up


\section{Limitations}


\section{Conclusion}


% Bibliography
\begin{thebibliography}{9}
\bibitem {origin}
A. Karpathy and L. Fei-Fei, Deep visual-semantic alignments for generating image descriptions,
In \textit{Proceedings of the IEEE Conference on Computer Vision and Pattern Recognition}, pages 3128-3137, 2015.

\end{thebibliography}

\end{document}

\documentclass[10pt,twocolumn,letterpaper]{article}

\usepackage{cvpr}
\usepackage[utf8]{inputenc}
\usepackage{gensymb}
\usepackage{graphicx}
\usepackage{caption}
\usepackage{subcaption}
\usepackage{mathtools}
\graphicspath{ {report3-imgs/} }
\usepackage{float}
\PassOptionsToPackage{hyphens}{url}\usepackage{hyperref}

\cvprfinalcopy
\def\cvprPaperID{a1700831}


% begin of document
\begin{document}
\title{Assignment 3 - Method for Generating Image Descriptions}
\author{Yuanzhong Xia\\
University of Adelaide\\
SA, Australia\\
{\tt\small a1700831@student.adelaide.edu.au}
}
\maketitle

% abstract
\begin{abstract}
% TODO: abstract
\end{abstract}

% content
\section{The Problem}
The problem is to give a sentence in natural language of the description for an input image.

This problem is very typical, because it is very challenging that it transforms information from ``image'' type to ``text'' type.
The input image should firstly be understood by the artificial intelligence,
then it should be able to output a descriptive text to tell what the input image shows.

The relative applications can be various, like: helping human to automatically pointing out the objectives and events in a massive number of images,
and helping visually impaired people to see. But the most difficult challenge is not only understanding both image and natural language,
but also the information transformation.


\section{Background}
Before this paper by Karpathy \textit{et al.} \cite{origin}. There are plenty of researches challenging this problem,
which can be mainly divided into three types.

\subsection{Image annotating strategy}
A typical way to annotate images is classifying or tagging images \cite{everingham}, \cite{russakovsky}.
In a fixed word dictionary, the image tagging can be more meaningful because unwanted words are discarded at the beginning.
The weakness is that the words in dictionary are often not rich enough.

Barnard \textit{et al.} \cite{barnard} and Socher \textit{et al.} \cite{socher}'s researches uses the images,
whose segments are annotated instead the whole image, as the training input data.
That helps the object recognition process, but requires more human works.

Gould \textit{et al.}'s research \cite {gould} used a method to extract scene feature,
so that the relationships between objects and the scene can be easily connected.
However, their work was focusing on labelling the scene, while the proposed method is used to
generate richer and high-level descriptions of the image, no matter what the scene labels are.


\subsection{Description generating strategy}
Methods from Kulkarni \textit{et al.} \cite{kulkarni} limit the resulting sentence number in one;
whereas this paper's authors think that manual restriction is not necessary and it reduces the artificial intelligence's creativity.

Matuszek \textit{et al.} \cite{matuszek} used grounding dependency tree which put words into a vector space.
The vector space would limit the results to some extend, like the sentence length, etc.


\subsection{Connecting between images and natural language strategy}
The earlier pioneering methods by Farhadi \textit{et al.} \cite{farhadi} hard-coded visual concepts and explicitly defined sentence.
Images objects or scenes are pre-defined, then marked images are learnt by the system so that the system can recognize defined things.
However, these methods limited the language variety and required much human works to define the connecting between language words and the images first.

Now, pre-trained image/word vectors are quite popular.
For images, ImageNet \cite{imagenet} is a very popular pre-trained image recognition model;
for words, previous works \cite{bengio}, \cite{socher2}, \cite{mikolov} mentioned the pre-trained word vectors
to obtain low-dimensional representations of words.
Additionally, in the proposed paper, the authors use these pre-trained models as well.


\section{Algorithm Description}
Unlike the previous approaches,


% TODO: how to tag more effectively (+5 times selected)


\section{Hypothesis}
% 1. two data model problem: one object in one model, not in another model (PS an object)
% 2. input generation size not enough, use small object with high resolution


\section{Experiments}
% details set up


\section{Limitations}


\section{Conclusion}


% Bibliography
\begin{thebibliography}{99}
\bibitem {origin}
A. Karpathy and L. Fei-Fei, Deep visual-semantic alignments for generating image descriptions,
In \textit{Proceedings of the IEEE Conference on Computer Vision and Pattern Recognition}, pp. 3128-3137, 2015.

\bibitem {kulkarni}
G. Kulkarni, et al., Baby talk: Understanding and generating simple image descriptions,
In \textit{Proc. IEEE Conf. Comput. Vis. Pattern Recog.}, pp. 1601-1608, 2011.

\bibitem {farhadi}
A. Farhadi, et al., Every picture tells a story: Generating sentences from images,
In \textit{Proc. 11th Eur. Conf. Comput. Vis.}, pp. 15-29, 2010.

\bibitem {everingham}
M. Everingham, L. Van Gool, C. K. I. Williams, J. Winn, and A. Zisserman, The Pascal visual object classes (VOC) challenge,
In \textit{Int. J. Comput. Vis.}, vol. 88, no. 2, pp. 303–338, 2010.

\bibitem {russakovsky}
O. Russakovsky, et al., Imagenet large scale visual recognition challenge,
In \textit{Int. J. Comput. Vis.}, vol. 115, no. 3, pp. 211–252, 2015.

\bibitem {barnard}
K. Barnard, P. Duygulu, D. Forsyth, N. De Freitas, D. M. Blei, and M. I. Jordan, Matching words and pictures,
In \textit{J. Mach. Learn. Res.}, vol. 3, pp. 1107–1135, 2003.

\bibitem {socher}
R. Socher and L. Fei-Fei, Connecting modalities: Semi-supervised segmentation and annotation of images using unaligned text corpora,
In \textit{Proc. IEEE Conf. Comput. Vis. Pattern Recog.}, pp. 966–973, 2010.

\bibitem {gould}
S. Gould, R. Fulton, and D. Koller, Decomposing a scene into geometric and semantically consistent regions,
In \textit{Proc. IEEE 12th Int. Conf. Comput. Vis.}, pp. 1–8, 2009.

\bibitem {matuszek}
C. Matuszek, N. FitzGerald, L. Zettlemoyer, L. Bo, and D. Fox, A joint model of language and perception for grounded attribute learning,
In \textit{Proc. 29th Int. Conf. Mach. Learn.}, pp. 1671–1678, 2012.

\bibitem {bengio}
Y. Bengio, H. Schwenk, J.-S. Senecal, F. Morin, and J.-L. Gauvain, Neural probabilistic language models,
In \textit{Innovations in Machine Learning. Berlin, Germany}, 2006.

\bibitem {socher2}
R. Socher, J. Pennington, and C. Manning, Glove: Global vectors for word representation,
In \textit{Proc. Empirical Methods Natural Language Process.}, pp. 1532–1543, 2014.

\bibitem {mikolov}
T. Mikolov, I. Sutskever, K. Chen, G. S. Corrado, and J. Dean, Distributed representations of words and phrases and their compositionality,
In \textit{Proc. Advances Neural Inf. Process. Syst.}, pp. 3111–3119, 2013.

\bibitem {imagenet}
J. Deng, W. Dong, R. Socher, L.-J. Li, K. Li, and L. Fei-Fei, ImageNet: A large-scale hierarchical image database,
In \textit{Proc. IEEE Conf. Comput. Vis. Pattern Recog.}, pp. 248–255, 2009.

\end{thebibliography}

\end{document}
